\documentclass{article}
\usepackage[utf8]{inputenc}
\usepackage[left=3cm,top=3cm,right=3cm,bottom=3cm]{geometry}
\newcommand*\chem[1]{\ensuremath{\mathrm{#1}}}
\usepackage{amsmath}
\usepackage{mathtools}

\usepackage{natbib}
\usepackage{graphicx}

\let\oldthebibliography\thebibliography
\let\endoldthebibliography\endthebibliography
\renewenvironment{thebibliography}[1]{
  \begin{oldthebibliography}{#1}
    \setlength{\itemsep}{0em}
    \setlength{\parskip}{0em}
}
{
  \end{oldthebibliography}
}

\begin{document}

\begin{center}
\Huge
Identificación y exfoliación de monocristales de calcogenuros de metales de transición unidimensionales.

\vspace{3mm}
\Large Jesús David Rincón Puche

\large
201126021

\Large José Alejandro Montaña Cortés
\large 


\vspace{2mm}
\Large
Director: Paula Giraldo Gallo\\


\normalsize
\vspace{2mm}

\today
\end{center}

\begin{abstract}
 En este proyecto se caracterizará las propiedades químicas de metales de calcogenuros de transición unidimensionales para poder estudiar, con capas finas y un bulk, 
\end{abstract}

\normalsize
\section{Introduction}


\section{State of the art}
\subsection{Experimental and theoretical studies}



\section{Objectives}
\subsection{General Objective}

To study how changing the filling in the extended Fermi Hubbard Hamiltonian affects SC and CDW, using the DMRG algorithm. 

\subsection{Specific Objectives}

\begin{enumerate}
    \item To develop a DMRG code and to test it by comparing the simulations to known results
    \item  To calculate, using charge gaps and quantum entanglement, critical points between SC and CDW while varying filling.
    \item To determine the parameter regimen values that optimizes SC.
\end{enumerate}

\section{Methodology}

As mentioned in sections 1 and 2, this work will be perfomed using DMRG. First initialization files created from MatLab and then uploaded to the university cluster to simulate the ground state for different values of $U$ and $V$ in the left zone of the diagram~\ref{phaseDiag}. From these simulations, charge gap, correlation functions and entanglement will be calculated. This information will be analyzed to determine the parameter regimen values that optimizes SC.

\section{Ethical considerations}

All the data that will be used in this project will be properly cited acknowledging the work done by other authors. Since this is a theoretical work, it does not need to go the Ethical Committe of the Science Faculty of the University.

\section{Chronogram}

\begin{table}[htb]
	\begin{tabular}{|c|cccccccccccccccc| }
	\hline
	Task $\backslash$ weeks & 1 & 2 & 3 & 4 & 5 & 6 & 7 & 8 & 9 & 10 & 11 & 12 & 13 & 14 & 15 & 16  \\
	\hline
	1 & X & X & X & X &   &   &   &  &  &   &   &   &   &   &   &   \\
	2 &   & X & X & X  & X & X &  &   &   &  &  &  &   &  &  &   \\
	3 &   &   &   &   &   & X  & X  & X & X  & X  & X  & X &  X &   &  &   \\
	4 &  &  &  &  &  &  &  & X & X & X &  X & X  &  X & X  &  X &   \\
	5 &   &   &   &   &  &   &   &   &  & &  X & X &  X & X & X &   \\
	6 &   &   &   &   &  &   &   &   &  &   &   & X &  X & X  & X & X  \\
	\hline
	\end{tabular}
\end{table}
\vspace{1mm}

\begin{itemize}
	\item Task 1: Literature survey on DMRG and Extended Fermi Hubbard Hamiltonian.
	\item Task 2: Development and testing of DMRG code 
	\item Task 3: Simulation of ground states for different fillings and values of $U$ and $V$
	\item Task 4: Calculation of gaps, correlations and entanglement based on the results of the simulations.
	\item Task 5: Determination of phase diagram for each filling.
	\item Task 6: Write final document and publishable article.
\end{itemize}

\section{Experts in the field}

\begin{itemize}
    \item Luis Quiroga Puello (Universidad de Los Andes)
    \item Edgar Patiño (Universidad de Los Andes)
    \item Jereson Silva (Universidad Nacional de Colombia)
\end{itemize}

\section*{Firma del Director}
\vspace{1.5cm}

\section*{Firma del Codirector}
\vspace{1.5cm}

\bibliographystyle{unsrt}
\bibliography{references}
\end{document}




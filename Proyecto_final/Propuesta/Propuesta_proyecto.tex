\documentclass{article}
\usepackage[utf8]{inputenc}
\usepackage[left=3cm,top=3cm,right=3cm,bottom=3cm]{geometry}
\newcommand*\chem[1]{\ensuremath{\mathrm{#1}}}
\usepackage{amsmath}
\usepackage{mathtools}

\usepackage{natbib}
\usepackage{graphicx}

\let\oldthebibliography\thebibliography
\let\endoldthebibliography\endthebibliography
\renewenvironment{thebibliography}[1]{
  \begin{oldthebibliography}{#1}
    \setlength{\itemsep}{0em}
    \setlength{\parskip}{0em}
}
{
  \end{oldthebibliography}
}

\begin{document}

\begin{center}
\Huge
Identificación y exfoliación de monocristales de calcogenuros de metales de transición unidimensionales.

\vspace{3mm}
\Large Jesús David Rincón Puche

\large
201126021

\Large José Alejandro Montaña Cortés
\large 


\vspace{2mm}
\Large
Director: Paula Giraldo Gallo\\


\normalsize
\vspace{2mm}

\today
\end{center}

\begin{abstract}
 En este proyecto se caracterizará las propiedades químicas de metales de calcogenuros de transición unidimensionales para poder estudiar, con capas finas y un bulk,la dependencia del corrimiento Raman en función del número de capas atómicas.
\end{abstract}

\normalsize
\section{Introducción}

La interacción de un campo electromagnético con la estructura electrónica es un proceso que se observa en muchas partes de la naturaleza, sin embargo, el entendimiento de estos fenómenos, permanece siendo aún un problema abierto para una gran cantidad de materiales que poseen espectros que difieren significativamente a los esperados teóricamente. Con el fin de hacer uso de nuevos materiales en distintos tipos de aplicaciones, resulta necesario tener modelos teóricos que predigan de una forma más exacta la respuesta de estos materiales a distintos tipos de interacción.\\\\

Un caso particular de estos materiales por estudiar son los calcogenuros de metales de transición. Estos son materiales interesantes para el estudio de propiedades tanto ópticas como eléctricas~\cite{dical}, dado que al cambiar su composición a través del dopaje se observan fenómenos físicos como superconductividad, Onda Densidad de Carga~\cite{CDW} (CDW, por sus siglas en inglés), entre muchos otros. En esta familia de materiales, se han encontrado algunos que presentan propiedades unidimensionales, y por lo mencionado anteriormente, es interesante desarrollar un estudio para entender de mejor manera las propiedades las propiedades químicas y electrónicas asociadas a estos materiales.

Por esta razón, el objetivo de este proyecto es realizar una caracterización química de estos materiales usando difracción de rayos X y espectroscopia de energía dispersa (EDS por sus siglas en inglés). Seguido de esto, se realizará una revisión bibliográfica sobre dichas propiedades. Una vez toda esta información esté conocida, se realizará una exfoliación para reducir hasta capas atómicas las muestras y por último realizar espectroscopia Raman para analizar la dependencia del corrimiento Raman de picos característicos con el número de capas atómicas.

\section{Estado del arte}
\subsection{Calcogenuros de metales de transición}



\section{Objetivos}
\subsection{Objetivo general}

Realizar una caracterización química y una espectroscopia para calcogenuros de metales de transición.

\subsection{Specific Objectives}

\begin{enumerate}
    \item Realizar la caracterización química de monocristales de calcogenuros de metales de transición unidimensionales por medio rayos x y EDS (Energy dispersive Spectroscopy).
    \item Reducir las capas de estos materiales unidimensionales a sus alturas mínimas (idealmente un par de capas atómicas) por medio de procesos de exfoliación mecánica.
    \item Caracterizar con microscopia de fuerza atómica y espectroscopia Raman tanto para el bulk como para las muestras exfoliadas, para conocer la dependencia del corrimiento Raman de picos característicos con el número de capas atómicas.
\end{enumerate}

\section{Metodología}

\section{Cronograma}

\begin{table}[htb]
	\begin{tabular}{|c|cccccccccccccccc|}
	\hline
	Task $\backslash$ weeks & 1 & 2 & 3 & 4 & 5 & 6 & 7 & 8 & 9 & 10&\\
	\hline
	1 & X & X & X & X &   &   &   &  &  &        \\
	2 &   &  & & X & X &  &  &   &   &   \\
	3 &   &   &   & X  &   &   &   & &   &    \\
	4 & & & & & & &X &X &X & &              \\
	5 &  &  &  &  &  &  &  X& X & X & X    \\
	6& & & & & & & & &X & X&               \\
	\hline
	\end{tabular}
\end{table}
\vspace{1mm}

\begin{itemize}
	\item Tarea 1: Caracterización con difracción de rayos X y EDS
	\item Tarea 2: Busqueda de bibliografía para propiedades físicas relacionadas con la caracterización.
	\item Tarea 3: Presentación con mitad de proyecto.
	\item Tarea 4: Exfoliación mecánica a materiales unidimensionales.
	\item Tarea 5: Espectroscopia Raman.
	\item Tarea 6: Escritura informe final y preparación del póster final. 

\end{itemize}

\section*{Firma del Director}
\vspace{1.5cm}

\section*{Firma del Codirector}
\vspace{1.5cm}

\bibliographystyle{unsrt}
\bibliography{references}
\end{document}




\documentclass[a4paper,10pt]{article}
\usepackage[spanish]{babel}
\usepackage[latin1]{inputenc}
%\usepackage[english]{babel}
%\addto{\captionsenglish}{\renewcommand{\refname}{Bibliograf\'{i}a}}
%\usepackage[utf8x]{inputenc}
\usepackage{graphicx,amsmath,enumerate,mathrsfs,amssymb,multicol,array}

\let\oldthebibliography\thebibliography
\let\endoldthebibliography\endthebibliography
\renewenvironment{thebibliography}[1]{
  \begin{oldthebibliography}{#1}
    \setlength{\itemsep}{0em}
    \setlength{\parskip}{0em}
}
{
  \end{oldthebibliography}
}

%opening
\title{\large{\textbf{T\'{i}tulo: Control de ondas de densidad de carga y superconductividad en sistemas fermi\'onicos por medio de forzamiento peri\'odico}}}

\date{}

\setlength{\textwidth}{6.6in}

\setlength{\textheight}{25.8cm}

\setlength{\marginparwidth}{0pt}

\setlength{\marginparsep}{0pt}

\setlength{\oddsidemargin}{0pt}

\setlength{\evensidemargin}{0pt}

\renewcommand{\baselinestretch}{1.2}


\hoffset-0.4cm

\voffset-2.6cm

\begin{document}

\maketitle

\setlength{\parindent}{0pt}

\textbf{Investigador principal:}\\
Dr. Juan Jos\'e Mendoza Arenas\\
Investigador Postdoctoral, Departamento de F\'{i}sica\\
Universidad de los Andes\\
Bogot\'a - Colombia\\

\textbf{Valor Total del proyecto:} $\$$39.893.700\\
\textbf{Valor Solicitado a la Fundaci\'on:} $\$$18.000.000

\newpage

\section{Resumen ejecutivo}

La superconductividad, caracterizada por el flujo de corriente el\'ectrica sin resistencia a trav\'es de un material, es una de las manifestaciones m\'as espectaculares de la f\'isica cu\'antica a nivel macrosc\'opico. Desde su descubrimiento se ha buscado su existencia a temperaturas cada vez m\'as altas, debido al enorme impacto tecnol\'ogico que tendr\'ia de sobrevivir en condiciones ambientales. A pesar de los grandes esfuerzos para alcanzar dicho objetivo, muchos interrogantes a\'un persisten sobre la naturaleza de la superconductividad en materiales complejos y las condiciones para su estabilizaci\'on. Un camino prometedor en esta empresa es la observaci\'on de respuestas superconductoras en sistemas correlacionados en condiciones fuera del equilibrio, inducidas al excitarlos con radiaci\'on. Aunque experimentos en esta direcci\'on han indicado posibles manifestaciones de superconductividad incluso a temperatura ambiente, los procesos f\'isicos que gobiernan esta din\'amica a\'un no han sido comprendidos. Un mecanismo que podr\'ia explicar estas observaciones es la supresi\'on controlada por radiaci\'on de fases que compiten con la superconductora, principalmente de la fase conocida como onda de densidad de carga. El presente proyecto busca realizar un estudio te\'orico de la din\'amica de ondas de densidad de carga bajo condiciones que simulan dicha excitaci\'on. El objetivo principal es determinar c\'omo se puede fortalecer o debilitar dicha fase, y si al ser suprimida emergen se\~nales de superconductividad. Para ello se propone un an\'alisis basado en el modelo de Fermi-Hubbard, que describe los procesos m\'as b\'asicos que experimentan los electrones en un material. Introduciendo una modulaci\'on peri\'odica para representar la excitaci\'on luminosa, se estudiar\'a la evoluci\'on temporal de estados de ondas de densidad de carga de diferentes maneras. Por un lado se aplicar\'a la teor\'ia de Floquet, que permite ver de forma general el impacto de la modulaci\'on peri\'odica en el comportamiento efectivo del modelo. Luego se simular\'a la evoluci\'on temporal de las ondas tanto en una dimensi\'on espacial (1D) como en el l\'imite de altas dimensiones. Se usar\'an los m\'etodos num\'ericos m\'as poderosos que existen para analizar la f\'isica fuera del equilibrio de cada caso: el grupo de renormalizaci\'on de la matriz densidad temporal (1D) y la teor\'ia de campo medio din\'amica (altas dimensiones). Inicialmente se buscar\'an evidencias de superconductividad por medio de funciones de correlaci\'on de pares. Sin embargo, dado que dicho estado es un efecto cu\'antico colectivo, es de esperar que tambi\'en sea detectada por una medida de cuantizaci\'on a nivel macrosc\'opico. Se estudiar\'a dicha medida, basada en desigualdades de Leggett-Garg, como una nueva forma de establecer la existencia de un estado superconductor. Con el estudio propuesto se espera determinar te\'oricamente si dicho estado surge fuera del equilibrio al suprimir la onda de densidad de carga competidora, e inspirar nuevos estudios experimentales de manipulaci\'on de sistemas correlacionados con luz.

\section{Planteamiento del problema}

Uno de los objetivos m\'as ambiciosos de la f\'isica moderna es estabilizar superconductividad a temperaturas mucho mayores que las logradas hasta ahora. Actualmente los materiales superconductores tienen una gran variedad de usos como la generaci\'on de campos para obtener im\'agenes por resonancia magn\'etica~\cite{aarnink2012euro} o para manipular part\'iculas en grandes aceleradores~\cite{bottura2016ieee}. Obtener esta fase a temperaturas cercanas a la del medio ambiente supondr\'ia un avance tecnol\'ogico revolucionario, pues extender\'ia ampliamente este espectro de aplicaciones. Permitir\'ia, por ejemplo, transportar corriente sin disipaci\'on a trav\'es de largas distancias o el uso masivo de medios de transporte de levitaci\'on magn\'etica~\cite{wang2009ieee}. Grandes pasos se han dado en el camino hacia tal objetivo, como el descubrimiento de los superconductores de alta temperatura~\cite{bednorz_muller} y la reciente observaci\'on de superconductividad a 203K en compuestos de hidr\'ogeno sometidos a altas presiones~\cite{drozdov2015nat}. Sin embargo, a\'un falta mucho por comprender, en particular sobre sistemas fuertemente correlacionados donde las interacciones entre electrones de conducci\'on juegan un papel crucial, y por tanto donde la teor\'ia convencional de la superconductividad (conocida como teor\'ia BCS~\cite{tinkham}) no es aplicable. Por ejemplo, el mecanismo responsable de la superconductividad de alta temperatura en materiales de cobre~\cite{bednorz_muller,keimer2015nat} a\'un est\'a sujeto a controversia, a pesar de haber sido estudiado por casi 30 a\~nos.  

Una de las razones por las cuales la descripci\'on de este tipo de superconductividad no convencional constituye un gran reto es que presenta una relaci\'on intrincada con otras fases de la materia que pueden emerger a su alrededor. Las interacciones fuertes entre part\'iculas dan lugar a estados con cierto tipo de orden magn\'etico o electr\'onico. Uno de los casos m\'as emblem\'aticos es el de ondas de densidad de carga (CDW por sus siglas en ingl\'es), que son modulaciones peri\'odicas de la densidad de electrones de conducci\'on, y que se pueden encontrar en una gran variedad de materiales. Una posible explicaci\'on del comportamiento de dichos materiales es que esta fase ordenada de carga compite con la fase superconductora, y es responsable de su desaparici\'on a las temperaturas cr\'iticas conocidas en equilibrio~\cite{gabovich2010acmp,chang2012nat}. No existe una visi\'on completa de esta competencia, cuyos detalles var\'ian de un material a otro. Sin embargo se ha sugerido que la supresi\'on de la fase CDW por alg\'un m\'etodo podr\'ia incrementar las temperaturas cr\'iticas superconductoras.

En la b\'usqueda de respuestas a \'este y otros interrogantes de la f\'isica de sistemas de muchas part\'iculas interactuantes, se han producido avances importantes para su estudio experimental en la \'ultima d\'ecada. Existen dos frentes principales de investigaci\'on. Primero, la implementaci\'on de sistemas cu\'anticos destinados a simular modelos de materia condensada y otras \'areas de la f\'isica, conocidos como simuladores cu\'anticos~\cite{georgescu2014rmp}, ha permitido entender una gran variedad de fen\'omenos en sistemas correlacionados. Segundo, el desarrollo de m\'etodos de control de diferentes grados de libertad en materiales complejos por medio de pulsos de luz ha permitido revelar f\'isica novedosa en configuraciones fuera del equilibrio, por medio de experimentos ``pump-probe" \cite{mankowski2016rep}. El ejemplo m\'as prominente es la observaci\'on de caracter\'isticas superconductoras en \'oxidos de cobre~\cite{hu2014nat} y fullerenos~\cite{mitrano2016nat} a temperaturas mayores que las obtenidas en equilibrio, y que parecen sobrevivir a temperatura ambiente. Este descubrimiento constituye la principal motivaci\'on del presente proyecto. 

Diversas explicaciones se han propuesto para estos resultados experimentales~\cite{hu2014nat,bukov2016prl,jonathan2016}. Una de las m\'as atractivas, en l\'inea con lo descrito anteriormente, corresponde a la supresi\'on de una fase ordenada que compite con la superconductora, como la CWD, debido a la excitaci\'on del material por medio de los pulsos de luz. Esta propuesta ha motivado bastante investigaci\'on tanto te\'orica~\cite{sentef2017prl} como experimental~\cite{forst2014prb,singer2016prl,mankowsky2017prl} durante los \'ultimos tres a\~nos sobre la respuesta de estados CDW a diferentes tipos de excitaci\'on \'optica. As\'i mismo se ha incentivado la implementaci\'on de fases CDW en simuladores cu\'anticos de \'atomos fr\'ios en redes \'opticas~\cite{messer2015prl}, en los cuales se han dado avances importantes en el an\'alisis de sistemas forzados peri\'odicamente~\cite{eckardt2017rmp,gorg2017}. Sin embargo estos estudios a\'un no han dado una respuesta definitiva a la pregunta del origen de las se\~nales superconductoras, y los tratamientos te\'oricos se han limitado al uso de teor\'ia de campo medio est\'atica~\cite{sentef2017prl}, la cual descarta completamente efectos de correlaciones entre part\'iculas. Por esto una investigaci\'on m\'as profunda es necesaria sobre este problema. En el presente proyecto se propone precisamente tal investigaci\'on a nivel te\'orico, basada en un modelo sencillo pero bastante rico de electrones interactuantes en una red, por medio de m\'etodos que s\'i tienen en cuenta efectos de correlaci\'on. 

Para tal fin, idealmente se desear\'ia una soluci\'on anal\'itica exacta de la din\'amica del modelo bajo consideraci\'on. Sin embargo, en general tal soluci\'on no es posible en casos donde diferentes fen\'omenos compiten con amplitudes similares, y por tanto donde ni siquiera se puede usar teor\'ia de perturbaciones. Una soluci\'on num\'erica exacta tampoco es viable, pues solo es posible para sistemas de muy pocas part\'iculas. Estas dificultades llevaron a la invenci\'on de los m\'etodos de redes de tensores~\cite{schollwock2011ann}, llamados tambi\'en m\'etodos TNT por sus siglas en ingl\'es, que por medio de un tratamiento adecuado de correlaciones cu\'anticas pueden describir correctamente propiedades est\'aticas y din\'amicas de sistemas de baja dimensionalidad de varios cientos de part\'iculas. Para altas dimensiones estos algoritmos tienen limitaciones fundamentales; sin embargo existe una aproximaci\'on altamente exitosa dise\~nada para este l\'imite, llamada teor\'ia de campo medio din\'amica o DMFT~\cite{georges1996rmp}. El uso de ambos tipos de m\'etodos en situaciones fuera del equilibrio~\cite{eckstein2014rmp} se ha extendido en los \'ultimos a\~nos por la urgencia de entender las variadas observaciones experimentales tanto en materiales forzados con radiaci\'on como en simuladores cu\'anticos. Es importante notar que ambos tipos de m\'etodos son requeridos para alcanzar una visi\'on completa de la f\'isica fuera del equilibrio de diferentes modelos, dado que en cada caso de dimensionalidad los fen\'omenos que gobiernan la din\'amica son muy diferentes~\cite{balzer2015prx}.  

En el presente proyecto se propone, entonces, desarrollar simulaciones num\'ericas basadas en algoritmos tipo TNT y DMFT fuera del equilibrio para estudiar la evoluci\'on temporal de estados CDW en sistemas de electrones interactuantes en diferentes dimensionalidades, y observar si a partir de ellos es posible inducir superconductividad por medio de forzamiento peri\'odico. Las preguntas concretas que se busca responder son

\begin{enumerate}

\item ?`C\'omo es la din\'amica de estados electr\'onicos CDW en 1D y altas dimensiones, en funci\'on de la amplitud de la interacci\'on repulsiva entre los electrones? ?`En qu\'e casos dichos estados son m\'as estables, y en qu\'e casos decaen m\'as r\'apido?

\item ?`C\'omo cambia la din\'amica del sistema cuando es objeto de forzamiento peri\'odico? ?`Para qu\'e tipos de excitaci\'on es posible fortalecer o suprimir estados CDW a comparaci\'on de la din\'amica sin forzamiento? 

\item En los casos en que el forzamiento peri\'odico permita suprimir los estados CDW m\'as r\'apidamente que en ausencia de tal forzamiento, ?`se observa comportamiento superconductor? En caso afirmativo, ?`de qu\'e formas se puede evidenciar el surgimiento de superconductividad? 

\end{enumerate}

La investigaci\'on propuesta en este proyecto constituye un paso necesario para comprender uno de los fen\'omenos m\'as emocionantes recientemente observados en f\'isica de materia condensada. Debido al alt\'isimo desarrollo tecnol\'ogico que tendr\'ia la implementaci\'on de sistemas superconductores a temperaturas cada vez m\'as altas, los resultados obtenidos ser\'ia de gran relevancia tanto a nivel nacional como global.

\section{Situaci\'on actual del conocimiento en el \'area de la investigaci\'on}

\subsection{Din\'amica en materiales complejos excitados con radiaci\'on}\label{cavalleri}
En la \'ultima d\'ecada se han desarrollado t\'ecnicas experimentales que permiten manipular materiales de part\'iculas fuertemente interactuantes a un nivel sin precedentes. En particular la irradiaci\'on de estos materiales con pulsos de l\'aser, de frecuencias principalmente en el infrarrojo medio, permite excitar a voluntad modos electr\'onicos y fon\'onicos~\cite{mankowski2016rep}. Dichas excitaciones inducen estados fuera del equilibrio donde emerge f\'isica novedosa y fascinante, observada al irradiar con un segundo pulso de prueba en THz (experimentos ``pump-probe"). 

El fen\'omeno m\'as prominente de los observados de esta manera es el de se\~nales superconductoras a temperaturas mayores que las temperaturas cr\'iticas en equilibrio $T_c$. Este efecto fue inicialmente reportado en YBa$_2$Cu$_3$O$_{6.5}$~\cite{hu2014nat}, un t\'ipico superconductor de alta temperatura, induciendo oscilaciones de algunos \'atomos de ox\'igeno (figura 1(a)). Luego se encontr\'o en el fullereno K$_3$C$_{60}$~\cite{mitrano2016nat} al alterar su estructura molecular por las excitaciones de luz (figura 1(b)). A pesar de la diferente naturaleza de estos materiales, se encontraron similitudes en su respuesta a la excitaci\'on. En ambos casos se midi\'o la conductividad \'optica $\sigma(\omega)=\sigma_1(\omega)+\sigma_2(\omega)$, donde un valor positivo de $\sigma_2(\omega)$ y una divergencia $\sigma_2(\omega)\sim1/\omega$ a baja frecuencia indican comportamiento superconductor. En el YBaCuO se encontr\'o un aumento transiente (7 ps) de la densidad superfluida para $T<T_c=60$K, y que la caracter\'istica superconductora sobreviv\'ia para $T>T_c$, llegando a estar presente hasta $310$K (figura 1(c)). Similarmente se observ\'o aparici\'on de superfluidez en el fullereno para $T>T_c=20$K. Aunque la existencia de superconductividad a temperatura ambiente no se ha establecido con certeza, estos resultados sugieren un camino interesante para lograr uno de los objetivos m\'as buscados de la f\'isica moderna.

\begin{figure}
\begin{center}
\includegraphics[scale=0.35]{./Figuras/Experimentos_Cavalleri.eps}
\caption{\label{exper_cavalleri} Excitaci\'on de materiales correlacionados por pulsos de luz. (a) Oscilaci\'on de \'atomos apicales de O en YBaCuO (puntos rojos). (b) Deformaci\'on de estructura de K$_3$C$_{60}$ por radiaci\'on. (c) Diferencia de conductividad \'optica de YBaCuO $\omega\Delta\sigma_2$ entre estados en equilibrio y fuera del equilibrio, en funci\'on de temperatura. (d) Esquema de un estado CDW, donde la acumulaci\'on de carga var\'ia de forma peri\'odica en el espacio~\cite{singer2016prl}. (e) Pico de difracci\'on de fase CDW en YBaCuO, y se\~nal superconductora transiente.}
\end{center}
\end{figure}

Se han propuesto diferentes explicaciones para estas observaciones. Para el YBaCuO se ha sugerido que la excitaci\'on coherente puede suprimir una fase que compite con la superconductora. La m\'as prominente de ellas es la CDW, que corresponde a una distribuci\'on espacial peri\'odica de regiones con poca y mucha carga y se esquematiza en la figura~\ref{exper_cavalleri}(d). De hecho se observ\'o experimentalmente que al inducir el estado superconductor con luz, al menos un $50\%$ del estado CDW se derret\'ia, llevando a un decaimiento del correspondiente pico de difracci\'on de rayos x (figura 1(e))~\cite{forst2014prb}. A pesar de esta evidencia, no existe una prueba definitiva del mecanismo. Desde entonces se han desarrollado nuevos experimentos enfocados en controlar la din\'amica de estados CDW por medio de radiaci\'on, ya sea para fortalecerlos~\cite{singer2016prl} o suprimirlos~\cite{mankowsky2017prl}. 

Debido a lo anterior, una comprensi\'on te\'orica de los procesos din\'amicos observados es requerida. En el presente proyecto se propone un camino, basado en el estudio de propiedades generales de sistemas fermi\'onicos interactuantes fuera del equilibrio. \'Este tiene la ventaja de que puede revelar f\'isica de aplicabilidad a una gran cantidad de sistemas donde dichas propiedades est\'an presentes, y que dicha f\'isica puede despu\'es ser corroborada experimentalmente en simuladores cu\'anticos~\cite{georgescu2014rmp}.

\subsection{Modelo te\'orico: Hamiltoniano de Fermi-Hubbard}

Para llevar a cabo el estudio propuesto, se considerar\'a la evoluci\'on de estados CDW en un sistema descrito por el modelo de Fermi-Hubbard~\cite{mahan}. \'Este determina la din\'amica de electrones interactuantes en una red, y a pesar de su aparente simplicidad contiene una f\'isica bastante rica. Como los electrones son fermiones de esp\'in $1/2$, tienen dos posibles orientaciones de n\'umero cu\'antico magn\'etico de esp\'in $\sigma$ $(\sigma=\uparrow,\downarrow)$, y solamente puede haber dos de ellos en un mismo sitio, con $\sigma$ opuesto. El modelo est\'a dado por el Hamiltoniano 
\begin{equation} \label{hami_hubbard}
H_{\rm Hub}(t) = -J_0\sum_{\langle i,j\rangle,\sigma} c_{i\sigma}^{\dagger}c_{j\sigma}+U\sum_i\left(n_{i\uparrow}-\frac{1}{2}\right)\left(n_{i\downarrow}-\frac{1}{2}\right)+\sum_if_{i\sigma}(\omega,t)n_{i\sigma}.
\end{equation}
Este Hamiltoniano incorpora tres tipos de operadores para cada sitio $i$: $c_{i\sigma}^{\dagger}$ $(c_{i\sigma})$ representa la creaci\'on (destrucci\'on) de un electr\'on con orientaci\'on de esp\'in $\sigma$, y $n_{i\sigma}=c_{i\sigma}^{\dagger}c_{i\sigma}$ es el operador n\'umero para esp\'in $\sigma$. El modelo tambi\'en incluye acoplamientos entre los operadores: $J_0>0$ es el par\'ametro de hopping, que determina la probabilidad de que un electr\'on salte de un sitio de red a un primer vecino (indicado por la notaci\'on $\langle i,j\rangle$); $U>0$ es la repulsi\'on de Coulomb, que se ha supuesto finita cuando dos electrones se encuentran en el mismo sitio de red; y $f_{i\sigma}(\omega,t)$ es una funci\'on peri\'odica en el tiempo con frecuencia $\omega$, que representa un forzamiento peri\'odico sobre el sistema. Los procesos descritos por este Hamiltoniano se bosquejan en la figura~\ref{hamiltoniano_y_cdw}. 

\begin{figure}
\begin{center}
\includegraphics[scale=1]{./Figuras/Hamiltonian_terms_2.eps}
\caption{\label{hamiltoniano_y_cdw} Esquema del modelo de Fermi-Hubbard. Los electrones en un sitio de red son representados por esferas, y las flechas indican la orientaci\'on de esp\'in. (a) Hopping: un electr\'on salta de un sitio a un vecino inmediato, caracterizado por el par\'ametro de hopping $J_0$. (b) Repulsi\'on de Coulomb: penaliza con energ\'ia $U$ el que haya dos electrones (de orientaci\'on de esp\'in contraria) en el mismo sitio. (c) Forzamiento peri\'odico $f(\omega)$ del potencial de cada sitio. (d) Representaci\'on de una CDW en una red 1D con periodicidad de dos sitios.}
\end{center}
\end{figure}

El modelo de Fermi-Hubbard sin forzamiento contiene la descripci\'on m\'as b\'asica de la competencia entre dos efectos: la propagaci\'on de electrones a lo largo de una red, y la repulsi\'on de Coulomb entre ellos. Si el primer proceso es dominante, es decir si $J_0\gg U$, los electrones se mueven de forma esencialmente libre a trav\'es de la red; en tal caso el sistema es met\'alico. Si domina la interacci\'on de Coulomb, de forma que $U\gg J_0$, dicha interacci\'on evita que los electrones se propaguen, formando un aislante de Mott. La inclusi\'on del potencial peri\'odico en el tiempo lleva a f\'isica a\'un m\'as rica, como se describe en la Secci\'on~\ref{sec_floquet}. Es importante resaltar que si $U<0$ se tiene un modelo de Fermi-Hubbard atractivo, pues hay un beneficio en energ\'ia por tener dos electrones en un mismo sitio de red. Este \'ultimo ha sido considerado un gran n\'umero de veces pues la atracci\'on efectiva entre electrones es la base de la superconductividad~\cite{tinkham}.  

Una gran cantidad de procesos f\'isicos han sido estudiados a partir del modelo~\eqref{hami_hubbard}, pues por su simpleza permite elucidar c\'omo emergen fen\'omenos generales de fermiones interactuantes. En el presente proyecto se propone estudiar c\'omo un estado inicial $|\psi(0)\rangle$ correspondiente a una CDW (como el de la figura~\ref{hamiltoniano_y_cdw}(d), que tiene periodicidad de dos sitios de red) evoluciona governado por un Hamiltoniano expl\'icitamente dependiente del tiempo, seg\'un la din\'amica establecida en la imagen de Schr\"odinger
\begin{equation} \label{evol_temp}
|\psi(t)\rangle=U(t,0)|\psi(0)\rangle\quad\text{con el propagador}\quad U(t,0)=\mathcal{T}\exp\left(-i\int_{t'}^{t}d\bar{t}H_{\rm Hub}(\bar{t})\right),
\end{equation}
y con $\mathcal{T}$ el operador de ordenamiento temporal. A pesar de que $H_{\rm Hub}$ descarta complicaciones como desorden, interacciones de largo alcance etc., el c\'alculo de su evoluci\'on temporal es bastante dif\'icil de realizar, en particular cuando $J_0\sim U$. Diferentes m\'etodos se han desarrollado a trav\'es de los a\~nos para hacerlo; a continuaci\'on se describen los elementos b\'asicos de aquellos que se utilizar\'an durante el proyecto.

\subsection{M\'etodos de an\'alisis}
Primero se discutir\'a una teor\'ia general para tratar sistemas con Hamiltoniano peri\'odico en el tiempo. Luego se describir\'an las ideas fundamentales de los algoritmos que se usar\'an para simular num\'ericamente la din\'amica de sistemas de muchos cuerpos gobernada por la ecuaci\'on~\eqref{evol_temp}. 

\subsubsection{Teor\'ia de Floquet} \label{sec_floquet}
El formalismo de Floquet describe de forma general el impacto que tiene el forzamiento peri\'odico en el tiempo en el comportamiento efectivo de un Hamiltoniano~\cite{bukov2015adv}. \'Este es equivalente temporal del teorema de Bloch para potenciales con periodicidad espacial. Para sistemas cu\'anticos cerrados, el teorema de Floquet establece que hay un conjunto completo de soluciones de la ecuaci\'on de Schr\"odinger dependiente del tiempo
\begin{equation} \label{schrodinger}
i\frac{d}{dt}|\psi(t)\rangle=H(t)|\psi(t)\rangle,\quad\text{con}\quad H(t+T)=H(t)\quad\text{y}\quad T=2\pi/\omega,
\end{equation}
de la forma $|\psi(t)\rangle=\exp(-i\varepsilon_{\alpha}t)|\psi_{\alpha}(t)\rangle$, con $|\psi_{\alpha}(t)\rangle=|\psi_{\alpha}(t+T)\rangle$ una funci\'on con la periodicidad del Hamiltoniano,
%\begin{equation}
%|\psi(t)\rangle=e^{-i\varepsilon_{\alpha}t}|\psi_{\alpha}(t)\rangle,\quad\text{con}\quad|\psi_{\alpha}(t)\rangle=|\psi_{\alpha}(t+T)\rangle,
%\end{equation}
donde las cuasi-energ\'ias $\varepsilon_{\alpha}$ yacen en la ``zona de Brillouin" $-\omega/2<\varepsilon_{\alpha}\leq\omega/2$. Expandiendo la funci\'on peri\'odica
%\begin{equation}
%|\psi_{\alpha}(t)\rangle=\sum_{m}e^{-im\omega t}|\psi_{\alpha,m}\rangle,
%\end{equation}
en t\'erminos de modos de Floquet $|\psi_{\alpha,m}\rangle$, se obtiene problema de eigenvalores
\begin{equation} \label{eigenvalue_problem}
(\varepsilon_{\alpha}+m\omega)|\psi_{\alpha,m}\rangle=\sum_{m'}H_{m-m'}|\psi_{\alpha,m'}\rangle,\quad\text{con}\quad H_m=\frac{1}{T}\int_0^Tdt\,e^{im\omega t}H_{\rm rot}(t),
\end{equation}
donde $H_{\rm rot}(t)$ es el Hamiltoniano en un marco de referencia rotante. Para el modelo de Fermi-Hubbard 1D dependiente del tiempo, es com\'un utilizar como forzamiento peri\'odico un campo el\'ectrico AC~\cite{tsuji2011prl,mentink2015nat}
\begin{equation} \label{field_ac}
f_{j\sigma}(\omega,t)=eaE_0j\sin(\omega t)=A\omega j\sin(\omega t),\qquad A = eaE_0/\omega, 
\end{equation}
con $e$ la carga del electr\'on, $a$ el par\'ametro de red, $E_0$ la amplitud del campo y $\omega$ la frecuencia de forzamiento. Esto resulta en componentes de Fourier del Hamiltoniano de la forma 
\begin{align} \label{fourier_hami_terms}
H_m=-J_0\sum_{\langle ij\rangle,\sigma}(-1)^m\mathcal{J}_m\Big((i-j)A\Big)c_{i\sigma}^{\dagger}c_{j\sigma}+\delta_{m,0}U\sum_{j}n_{j\uparrow}n_{j\downarrow},
\end{align}
donde la repulsi\'on de Coulomb solamente est\'a presente en el sector $m=0$, y el hopping queda renormalizado por la funci\'on de Bessel $\mathcal{J}_m\left((i-j)A\right)$. Truncando el n\'umero de sectores de Floquet, o tomando ciertos l\'imites espec\'ificos, es posible resolver el problema de eigenvalores~\eqref{eigenvalue_problem} resultante. Por ejemplo, a alta frecuencia $\omega\gg U,J_0$, los diferentes sectores de Floquet est\'an bastante separados en energ\'ia, como lo indica la ecuaci\'on~\eqref{eigenvalue_problem}, y el sistema se puede describir solamente considerando el sector $m=0$. Esto corresponde a un modelo de Fermi-Hubbard con un par\'ametro de hopping efectivo $J_0\mathcal{J}_0(A)$. Como $|\mathcal{J}_0(A)|<1$, el forzamiento peri\'odico induce una reducci\'on de la propagaci\'on de electrones a trav\'es del sistema. Incluso para amplitudes $A$ tales que $\mathcal{J}_0(A)=0$, el hopping se suprime por completo y se congela la din\'amica. Este fen\'omeno ya ha sido observado experimentalmente en simuladores cu\'anticos de \'atomos fr\'ios~\cite{eckardt2017rmp}. Para forzamientos tales que $\mathcal{J}_0(A)<0$ se cambia el signo de los procesos de hopping, lo que equivale a cambiar el signo de la interacci\'on de Coulomb. Es decir, se puede inducir una interacci\'on efectivamente atractiva entre electrones~\cite{eckstein2014rmp,tsuji2011prl}. Otros esquemas de forzamiento peri\'odico m\'as complicados han sido utilizados en el modelo de Fermi-Hubbard repulsivo~\eqref{hami_hubbard}, para los cuales la aplicaci\'on de la teor\'ia de Floquet ha permitido encontrar indicios de atracciones atractivas~\cite{bukov2016prl,nosotros2017driven} e incluso de superconductividad~\cite{jonathan2016}. El uso de esta teor\'ia es por tanto de vital importancia para el proyecto.

\subsubsection{Aproximaci\'on de campo medio est\'atica} \label{campo_medio}
Cuando se enfrenta un problema de muchas part\'iculas interactuantes, usualmente se desarrolla una soluci\'on aproximada simple antes de utilizar m\'etodos anal\'iticos o num\'ericos complejos. Tal soluci\'on es usualmente provista por la teor\'ia de campo medio est\'atica, en la cual 
%cada operador en el Hamiltoniano es reemplazado por su valor medio m\'as fluctuaciones. Por ejemplo, $n_{j\sigma}\approx\langle n_{j\sigma}\rangle+\delta n_{j\sigma}$, donde $\langle n_{j\sigma}\rangle=\langle\psi|n_{j\sigma}|\psi\rangle$ es el valor esperado del operador $n_{j\sigma}$ en el estado de inter\'es $|\psi\rangle$, y $\delta n_{j\sigma}$ determina sus fluctuaciones.
el sistema se reduce a un sitio de red acoplado a un campo efectivo creado por los dem\'as, eliminando por completo las correlaciones espaciales. Implementar esta aproximaci\'on transforma el problema original en uno mucho m\'as simple de solucionar, y en varios casos proporciona informaci\'on cualitativa valiosa. Sin embargo, es conocido que no proporciona informaci\'on correcta sobre las propiedades de sistemas cu\'anticos cerca de transiciones de fase, o en situaciones fuera del equilibrio.  

Hasta el momento, \'este ha sido el \'unico m\'etodo usado para estudiar te\'oricamente la competencia entre estados CDW y superconductores en sistemas con forzamiento peri\'odico~\cite{sentef2017prl}. En un modelo de Fermi-Hubbard atractivo se observ\'o que seg\'un el tipo de forzamiento, era posible suprimir la fase CDW en favor de la superconductora, o inducir el efecto contrario. Este estudio constituye una motivaci\'on importante para el presente proyecto, el cual busca ir m\'as all\'a en los siguientes aspectos. Primero, es necesario partir de un modelo repulsivo, dada la naturaleza de las interacciones electr\'onicas en estados normales. Segundo, para superar los problemas de la teor\'ias de campo medio est\'atica, se deben incluir correlaciones cu\'anticas en el sistema. A contiuaci\'on se describen las ideas m\'as b\'asicas de los m\'etodos que se usar\'an para hacerlo.

\subsubsection{Grupo de renormalizaci\'on de la matriz densidad}
El m\'etodo m\'as directo para calcular propiedades de un sistema de muchos cuerpos consiste en diagonalizar num\'ericamente su Hamiltoniano. Sin embargo, esto no es posible en general pues su tama\~no crece exponencialmente con el n\'umero de sitios, y su diagonalizaci\'on requiere un esfuerzo computacional demasiado grande. 

Esta dificultad se puede tratar con m\'etodos de renormalizaci\'on, que se enfocan en solucionar un problema efectivo que contiene la informaci\'on m\'as importante del problema original, y que por tanto es m\'as f\'acil de resolver num\'ericamente. El principal m\'etodo de esta familia es el grupo de renormalizaci\'on de la matriz densidad~\cite{white1992prl} (DMRG por sus siglas en ingl\'es), el cual supuso un gran avance en el entendimiento de sistemas de muchos cuerpos de baja dimensionalidad~\cite{schollwock2011ann}. Este algoritmo usa los estados propios de mayores valores propios de la matriz densidad reducida del sistema para determinar la informaci\'on m\'as probable que define el problema efectivo. De esta manera ha permitido analizar diversas redes 1D de hasta miles de sitios.

El DMRG fue inicialmente propuesto para estudiar estados de baja energ\'ia~\cite{white1992prl}. Posteriormente fue extendido para calcular evoluci\'on temporal~\cite{vidal2004prl}. Esta extensi\'on, denominada grupo de renormalizaci\'on de la matriz temporal o t-DMRG, se bas\'o en la observaci\'on de que cualquier estado $|\psi\rangle$ de un sistema cu\'antico cerrado se puede escribir como un producto de tensores, cada uno asociado a un sitio de red. Esta estructura, dada por
\begin{equation}
|\psi\rangle=\sum_{n_1,n_2,\ldots,n_N=1}^dA_1^{n_1}A_2^{n_2}\ldots A_N^{n_N}|n_1n_2\ldots n_N\rangle,
\end{equation} 
donde $d$ es la dimensi\'on del espacio de Hilbert de un sitio (para fermiones, $d=4$) y $A_j^{n_j}$ es el tensor $n_j$ del sitio $j$, es una representaci\'on compacta del estado, y es el ejemplo m\'as b\'asico de las llamadas redes de tensores. Escribir el estado $|\psi\rangle$ de esta manera permite manipularlo de forma eficiente. Adem\'as lleva a reformular el DMRG original de una forma natural~\cite{schollwock2011ann}. Por esto, ambos m\'etodos de renormalizaci\'on de matriz densidad (est\'atico y temporal) son los ejemplos principales de la teor\'ia de redes de tensores (TNT).

Muchos trabajos recientes han usado t-DMRG para estudiar la din\'amica del modelo de Fermi-Hubbard 1D bajo diversas condiciones. Dos en particular son relevantes para el presente proyecto. El primero corresponde a la simulaci\'on de la din\'amica de una red sin forzamiento que inicialmente se encuentra en un estado antiferromagn\'etico~\cite{heidrich_meisner2015pra}, e ilustra el mecanismo de demagnetizaci\'on a altas interacciones $U$, que corresponde a una din\'amica gobernada por interacci\'on de intercambio.
%Un mecanismo similar podr\'ia jugar un papel determinante la relajaci\'on de un estado CDW. 
El segundo realiza un an\'alisis del forzamiento de estados base y t\'ermicos de un Hamiltoniano fermi\'onico a un cuarto de llenado, y muestra c\'omo el forzamiento permite aumentar el acoplamiento atractivo entre electrones y por tanto las correlaciones superconductoras~\cite{jonathan2016}.

M\'as all\'a de 1D pueden surgir diferentes fen\'omenos. Por ejemplo, el mecanismo de intercambio responsable de la demagnetizaci\'on de cadenas 1D no es dominante en altas dimensiones~\cite{balzer2015prx}. Entonces, para tener una perspectiva amplia de la din\'amica de estados CDW, se requiere ir m\'as all\'a de una dimensi\'on. Existen razones fundamentales por las cuales los algoritmos TNT desarrollados hasta ahora no son \'utiles en tales casos~\cite{schollwock2011ann}. Por lo tanto es necesario recurrir a un m\'etodo diferente, que se describe brevemente a continuaci\'on.

\subsubsection{Teor\'ia de campo medio din\'amica fuera del equilibrio}
Uno de los m\'etodos m\'as poderosos para estudiar sistemas correlacionados de alta dimensionalidad es la teor\'ia de campo medio din\'amica (DMFT)~\cite{georges1996rmp}. Su idea consiste en reemplazar el sistema por una impureza acoplada con un campo medio $\Lambda$, como se esboza en la figura~\ref{dmft_scheme}(a). Sin embargo, a diferencia de la teor\'ia est\'atica, este campo ahora depende del tiempo e intercambia part\'iculas con la impureza, lo cual resulta en fluctuaciones din\'amicas que dan cuenta de efectos de correlaci\'on. La figura~\ref{dmft_scheme}(b) ilustra algunas de estas fluctuaciones en la ocupaci\'on de la impureza. Este m\'etodo ha sido recientemente extendido a f\'isica fuera del equilibrio~\cite{eckstein2014rmp}, motivado en gran medida por los avances experimentales en manipulaci\'on de sistemas de muchos cuerpos.

Existen varias maneras de solucionar el problema efectivo de impureza. Una de las m\'as exitosas consiste en discretizar el campo, transform\'andolo en un conjunto de sitios acoplados \'unicamente con la impureza~\cite{gramsch2014prb} (figura~\ref{dmft_scheme}(c)). Simulando la din\'amica de este sistema discreto con t-DMRG~\cite{alexander2014prb}, se observ\'o que la demagnetizaci\'on de un estado antiferromagn\'etico en altas dimensiones y $U\gg J_0$ no es gobernado por interacciones de intercambio (como en 1D) sino por el traspaso de energ\'ia de grados de libertad de carga a los de esp\'in~\cite{balzer2015prx}.
%y que si $U\sim J_0$ el proceso se debe a la destrucci\'on de los momentos magn\'eticos locales.
El mismo m\'etodo fue usado para demostrar c\'omo manipular el orden magn\'etico de sistemas fermi\'onicos con forzamiento peri\'odico~\cite{nosotros2017driven}. Adem\'as existen propuestas de implementaci\'on experimental del algoritmo en simuladores cu\'anticos~\cite{juha2016sci}. Debido a su enorme capacidad para estudiar sistemas de muchos cuerpos, y a su formulaci\'on en t\'erminos de m\'etodos TNT, se utilizar\'a el DMFT para estudiar la din\'amica de estados CDW en altas dimensiones, y el surgimiento de se\~nales superconductoras debido a un potencial peri\'odico en el tiempo.

\begin{figure}[t]
\begin{center}
\includegraphics[scale=0.25]{Figuras/DMFT_scheme.eps}
\caption{\label{dmft_scheme} DMFT fuera del equilibrio. (a) Una red correlacionada de altas dimensiones es transformada en un sistema de una impureza acoplada con un campo medio $\Lambda_{\sigma}(t,t')$ dependiente del tiempo. (b) Debido al intercambio de part\'iculas entre ambos, la ocupaci\'on de la impureza fluct\'ua en el tiempo. (c) En el modelo efectivo se puede discretizar el campo medio, donde los sitios de red que lo representan (esferas azules) solamente est\'an acoplados con la impureza (esfera verde). Figura modificada de Ref.~\cite{juha2016sci}.} 
\end{center}
\end{figure}

\subsection{Desigualdades de Leggett-Garg}
Existen diversas cantidades que pueden evidenciar la aparici\'on de superconductividad, como funciones de correlaci\'on, factores de estructura o par\'ametros de orden~\cite{jonathan2016,sentef2017prl}. Una forma alternativa corresponde a una medida de cuantizaci\'on a nivel macrosc\'opico, dada por las desigualdades de Leggett-Garg~\cite{lgi_original,emary2014rpp}. Dicha medida surge de considerar dos caracter\'isticas que un objeto macrosc\'opico cl\'asico debe cumplir: realismo macrosc\'opico (el objeto no se encuentra en una superposici\'on de estados) y mensurabilidad no invasiva (una medici\'on sobre el objeto no afecta su estado). Un sistema que satisface estas condiciones cumple que
\begin{equation} \label{lgi}
C(t_1,t_3)-C(t_1,t_2)-C(t_2,t_3)\geq-1,\qquad C(t_i,t_j)=\langle\{Q(t_i),Q(t_j)\}\rangle/2,
\end{equation}
con $C(t_i,t_j)$ las correlaciones temporales entre tiempos $t_i,t_j$ de una variable dicot\'omica $Q$ (que toma valores $\pm1$ en el caso cl\'asico). Por tanto la violaci\'on de estas desigualdades indica comportamiento no cl\'asico. Como un estado superconductor corresponde a un estado cu\'antico macrosc\'opico colectivo, y se ha encontrado que la violaci\'on de las desigualdades permite localizar transiciones de fase cu\'anticas~\cite{nosotros2016LGI}, resulta natural formular la hip\'otesis de que el surgimiento de superconductividad en la din\'amica de estados CDW puede ser evidenciado por dichas violaciones.
%Adem\'as el c\'alculo de correlaciones temporales se puede realizar de forma directa a partir de t-DMRG~\cite{nosotros2016LGI} y de DMFT~\cite{gramsch2014prb,alexander2014prb}.
Para la parte final del presente proyecto se propone verificar esta hip\'otesis. 

\subsection{Hip\'otesis de la propuesta de investigaci\'on}
Es conocido que para sistemas tipo Fermi-Hubbard, la relajaci\'on de estados iniciales antiferromagn\'eticos est\'a determinada por diferentes mecanismos que dependen de la dimensionalidad y la interacci\'on $U/J_0$~\cite{heidrich_meisner2015pra,balzer2015prx}. Similarmente se propone en el presente proyecto que estas condiciones tienen un impacto fundamental en la din\'amica de estados CDW, las cuales se busca determinar. Se establece la hip\'otesis de que por medio de forzamiento peri\'odico es posible manipular de forma controlada la evoluci\'on de estos estados, fortaleci\'endolos o suprimi\'endolos. Se sugiere que se pueden inducir estados superconductores al suprimir las configuraciones CDW, y que indicaciones de tales estados se pueden detectar de diferentes maneras, sean las usualmente utilizadas (basadas en correlaciones de pares) o una novedosa basada en desigualdades de Leggett-Garg.

\section{Objetivos de la investigaci\'on}

El proyecto descrito en la presente propuesta busca alcanzar el siguiente objetivo general, y los subsecuentes objetivos espec\'ificos para lograrlo.\\

\textbf{Objetivo general}\\
Determinar si es posible estabilizar superconductividad en sistemas fermi\'onicos interactuantes al manipular la fase CDW competidora por medio de un potencial peri\'odico en el tiempo, y encontrar las condiciones que maximizan dicha respuesta superconductora.\\

\textbf{Objetivos espec\'{i}ficos}

\begin{enumerate}
\item Simular la din\'amica de un estado inicial de tipo CDW en redes de Fermi-Hubbard sin forzamiento en 1D y altas dimensiones, y en cada caso determinar los mecanismos f\'isicos responsables de la relajaci\'on del estado para interacciones peque\~nas ($U<J_0$), intermedias ($U\sim J_0$) y altas ($U>J_0$). 
\item Determinar el impacto que tiene, en la din\'amica de cada caso, un forzamiento dependiente del tiempo simulado como un potencial peri\'odico~\eqref{field_ac}. En particular considerar forzamiento de alta frecuencia ($\omega\gg U,J_0$), resonante ($U=l\omega\gg J_0$ para un entero $l$) y no resonante de baja frecuencia ($J_0<\omega<U$), pues es conocido que cada uno afecta de forma muy diferente la evoluci\'on temporal~\cite{bukov2016prl,jonathan2016,tsuji2011prl,mentink2015nat,nosotros2017driven}.  
\item Identificar los tipos de forzamiento que pueden reforzar o destruir de forma m\'as r\'apida los estados CDW.
\item Calcular funciones de correlaci\'on y factores de estructura de pares en las condiciones de mayor supresi\'on de estados CDW, y establecer si se induce un estado superconductor.
\item Determinar c\'omo por medio de las desigualdades de Leggett-Garg se registran los cambios en la naturaleza del estado del sistema forzado, y si es posible indicar la aparici\'on de superconductividad.

\end{enumerate}

\section{Metodolog\'{i}a propuesta}
La investigaci\'on propuesta en el presente proyecto es puramente te\'orica, de manera que se enfoca en la realizaci\'on de simulaciones num\'ericas y c\'alculos anal\'iticos y no en el desarrollo de un proceso experimental o un trabajo de campo. 

La mayor parte del proyecto corresponde a simulaciones de din\'amica de estados CDW. Dichas simulaciones ser\'an realizadas en programas escritos en el lenguaje C, y estar\'an basadas en la librer\'ia TNT (Tensor Network Theory)~\cite{tnt}. \'Esta es una librer\'ia de acceso libre en l\'inea, enfocada en la implementaci\'on de m\'etodos de la familia TNT por medio de operaciones eficientes sobre tensores. Debido a su alto rendimiento, ser\'a utilizada para obtener la evoluci\'on temporal tanto de los sitemas 1D como del problema efectivo de impureza del m\'etodo DMFT para altas dimensiones. Para facilidad de manipulaci\'on de informaci\'on por parte del usuario, la librer\'ia TNT brinda la posibilidad de usar archivos de inicializaci\'on y de salida en formato de MATLAB. Todos los archivos de salida obtenidos en este formato ser\'an almacenados para realizar los an\'alisis requeridos. Se escribir\'an otros archivos en MATLAB para leer estos archivos de salida, recopilar los datos obtenidos directamente de las simulaciones, llevar a cabo los c\'alculos mencionados a continuaci\'on con dichos datos, y realizar las figuras requeridas.     

Los posibles problemas que se pueden presentar en el desarrollo de este proyecto se derivar\'ian de la enorme capacidad computacional que podr\'ia ser requerida para estudiar unas combinaciones particulares de acoplamientos en el modelo de Fermi-Hubbard, frecuencias $\omega$ y amplitudes $A$ de forzamiento. Las estrategias para solucionar estos problemas depender\'ian de los casos particulares. Sin embargo la librer\'ia TNT cuenta con par\'ametros de control que se pueden ajustar (paso de tiempo, errores de truncamiento en la evoluci\'on temporal, tolerancias para transformar en ceros valores num\'ericos muy peque\~nos, entre otros), y que permiten realizar un c\'alculo con menor esfuerzo computacional sin incurrir en errores importantes. De ser necesario se examinar\'an a fondo combinaciones de estos par\'ametros de control que permitan hacer las simulaciones m\'as demandantes de la forma m\'as eficiente posible en las facilidades computacionales disponibles. 

A continuaci\'on se describe la metodolog\'ia para alcanzar los objetivos espec\'ificos descritos anteriormente.

\begin{enumerate}
\item Primer objetivo: Primero se escribir\'an y probar\'an los programas para realizar evoluci\'on temporal en 1D (t-DMRG) y en altas dimensiones (DMFT). Luego se tomar\'an diferentes estados iniciales $|\psi(0)\rangle$ de tipo CDW, incluyendo el mostrado en la figura~\ref{hamiltoniano_y_cdw}(e) y estados similares sin doble ocupaci\'on perfecta en ning\'un sitio (dando as\'i lugar a la posibilidad de incrementar el orden de carga~\cite{singer2016prl}). Para cada estado se realizar\'a la evoluci\'on temporal~\eqref{evol_temp} manteniendo $J_0=1$ y barriendo sobre valores de $U$ en el rango $0\leq U\leq40$, considerando as\'i los diferentes escenarios de interacciones repulsivas entre electrones. El tama\~no de la muestra, que corresponde al n\'umero de valores de $U$ que se tomar\'an en este rango (para el presente y los dem\'as objetivos) se determinar\'a de forma tal que la f\'isica en los escenarios de interacciones d\'ebiles y fuertes sean completamente identificados. Inicialmente se considerar\'an pasos de $\Delta U=1$ para $U\leq10$, y de $\Delta U=5$ para $U>10$. De no ser suficiente se tomar\'an pasos $\Delta U$ m\'as peque\~nos. Se caracterizar\'a la din\'amica de cada caso calculando las dobles ocupaciones de cada sitio $j$ y la doble ocupaci\'on total para $N$ sitios de red, 
\begin{equation} \label{doble_oc}
d(t)=\frac{1}{N}\sum_jd_j(t)=\frac{1}{N}\sum_j\langle\psi(t)|n_{j\uparrow}n_{j\downarrow}|\psi(t)\rangle,
\end{equation}
y el imbalance entre la ocupaci\'on de sitios pares e impares,
\begin{equation} \label{imbalance}
I(t)=\frac{1}{N}\sum_{j\text{ impar}}I_j(t)=\frac{1}{N}\sum_{j\text{ impar}}\langle\psi(t)|n_{j}-n_{j+1}|\psi(t)\rangle,\quad\text{con}\quad n_j=n_{j\uparrow}+n_{j\downarrow}.
\end{equation}
Se comparar\'an los resultados con teor\'ia de perturbaciones para $U\gg J_0$ y con una soluci\'on anal\'itica para el caso integrable $U=0$. Luego se determinar\'an los mecanismos f\'isicos responsables de la relajaci\'on del estado CDW analizando las escalas de tiempo que gobiernan la din\'amica, de forma similar a la realizada en las referencias~\cite{balzer2015prx,nosotros2017driven,heidrich_meisner2015pra}. 

\item Segundo objetivo: Se incluir\'a en los programas el forzamiento dependiente del tiempo dado por la ecuaci\'on~\eqref{field_ac} (el potencial equivalente en altas dimensiones est\'a dado en~\cite{nosotros2017driven}). Luego para casos representativos de cada tipo de din\'amica revelada se calcular\'a la evoluci\'on temporal con diversos valores de $A$ y $\omega$, para implementar cada clase de excitaci\'on mencionada: alta frecuencia, resonancia y baja frecuencia fuera de resonancia. Se calcular\'an los valores esperados de las ecuaciones~\eqref{doble_oc} y~\eqref{imbalance}, y se contrastar\'an con los resultados en ausencia de forzamiento. Tambi\'en se comparar\'an con las predicciones generales de la teor\'ia de Floquet para cada tipo de excitaci\'on~\cite{bukov2016prl,tsuji2011prl,mentink2015nat}.

\item Tercer objetivo: Analizando todos los resultados obtenidos, se determinar\'an las condiciones en las cuales $d(t)$ e $I(t)$ crecen y decrecen en el tiempo de forma m\'as r\'apida. El crecimiento corresponde a un fortalecimiento de la fase CDW, mientras que el decaimiento corresponde a su supresi\'on.

\item Cuarto objetivo: En la din\'amica de m\'axima supresi\'on del estado CDW por medio de forzamiento, se buscar\'a una respuesta superconductora de la forma tradicional~\cite{jonathan2016}. \'Esta corresponde al an\'alisis de funciones de correlaci\'on de pares entre dos sitios $i$ y $j$,
\begin{equation}
P_{ij}(t)=\langle\psi(t)|\Delta_i^{\dagger}\Delta_j|\psi(t)\rangle=\langle\psi(t)|c_{i\uparrow}^{\dagger}c_{i\downarrow}^{\dagger}c_{j\downarrow}c_{j\uparrow}|\psi(t)\rangle,
\end{equation}
definidas a partir de los operadores de creaci\'on de pares de fermiones en un sitio cualquiera $j$, $\Delta_{j}^{\dagger}=c_{j\uparrow}^{\dagger}c_{j\downarrow}^{\dagger}$, y del correspondiente factor de estructura para un momento $q$,
\begin{equation}
P_q(t)=\frac{1}{N}\sum_{j,k}\langle\psi(t)|\Delta_j^{\dagger}\Delta_k|\psi(t)\rangle e^{iq(j-k)}.
\end{equation}
Un estado superconductor corresponde a un pico agudo en el diagrama de $P(q)$ vs $q$, que indica cuasi-condensaci\'on de pares en el estado de cuasi-momento $q$. Dependiendo de la localizaci\'on del pico se tiene diferente tipo de superconductividad~\cite{kitamura2016prb}. El pico en $q=0$ indica condensaci\'on de doblones en el estado de momento $q=0$, lo que se conoce como superconductividad ``s-wave" (como en teor\'ia BCS~\cite{tinkham}). El pico en $q=\pi$ indica condensaci\'on en un estado de m\'aximo momento en la zona de Brillouin, conocida como superconductividad $\eta$. La condensaci\'on en otros valores finitos de $q$ se conoce como superconductividad Fulde-Ferrell-Larkin-Ovchinnikov (FFLO). Se determinar\'a si en la din\'amica emerge un pico agudo en el factor de estructura, y el tipo de superconductividad de los mencionados anteriormente en caso afirmativo. 

\item Quinto objetivo: Para analizar las desigualdades de Leggett-Garg, se requieren las correlaciones temporales locales de un sitio, de la forma
\begin{equation} \label{correl_temp}
C_{j\sigma}(t,t')=-i\langle\mathcal{T} n_{j\sigma}(t)n_{j\sigma}(t')\rangle,\quad\text{con}\quad n_{j\sigma}(t)=U(0,t)n_{j\sigma}U(t,0),
\end{equation}
donde $U$ es el propagador de la ecuaci\'on~\eqref{evol_temp}, y donde el operador para el que se calculan las correlaciones es tal que en el l\'imite cl\'asico tiene solamente dos valores (en este caso, $n_{j\sigma}=0,1$). En el m\'etodo de t-DMRG para cadenas 1D, este c\'alculo implica realizar dos evoluciones temporales, una por cada operador $U$. En DMFT, por otro lado, correlaciones temporales de la forma $-i\langle\mathcal{T} c_{j\sigma}(t)c_{j\sigma}^{\dagger}(t')\rangle$ son necesarias durante el proceso de convergencia del m\'etodo~\cite{gramsch2014prb}. Debido a esto la extensi\'on para obtener las correlaciones~\eqref{correl_temp} es inmediata, y su c\'alculo se incluir\'a durante las simulaciones para el segundo objetivo. Se obtendr\'an las correlaciones temporales tanto en 1D como en altas dimensiones para los diferentes par\'ametros del Hamiltoniano estudiados, y se usar\'an para verificar si las desigualdades~\eqref{lgi} se violan. En caso afirmativo, se determinar\'a si esto ocurre en regiones donde se haya registrado comportamiento superconductor. De esta forma se establecer\'a si las desigualdades de Leggett-Garg constituyen una herramienta \'util para observar el surgimiento de nuevos estados en configuraciones fuera del equilibrio.

\end{enumerate}

\section{Cronograma}

El cronograma de actividades se presenta en detalle en un documento adjunto. 

\newpage

\section{Presupuesto}

\textbf{Presupuesto global}

\begin{center}
\begin{tabular}{|l|m{4cm}|r|r|}\hline
&\multicolumn{2}{c|}{\textbf{Fuente}}&\\ \cline{2-3} 
\textbf{Rubro} & \textbf{Fundaci\'on del Banco de la Rep\'ublica} & \textbf{Contrapartida} & \textbf{Total}\\ \hline 
Equipos &\multicolumn{1}{r|}{0}&0&0\\ \hline
Materiales e insumos &\multicolumn{1}{r|}{0}&0&0\\ \hline
Personal &\multicolumn{1}{r|}{9.000.000}&15.953.700&24.953.700\\ \hline
Viajes &\multicolumn{1}{r|}{9.000.000}&0&9.000.000\\ \hline 
Publicaciones &\multicolumn{1}{r|}{0}&0&0\\ \hline
Costos administrativos&\multicolumn{1}{r|}{0}&5.940.000&5.940.000\\ \hline
TOTAL&\multicolumn{1}{r|}{18.000.000}&21.893.700&39.893.700\\ \hline
\end{tabular}
\end{center}

\textbf{Presupuesto detallado para el personal}

\begin{center}
\begin{tabular}{|m{1.9cm}|m{1.9cm}|m{2cm}|m{2.6cm}|m{3.7cm}|l|}\hline
&&&&\multicolumn{2}{c|}{\textbf{Fuente}}\\ \cline{5-6} 
\textbf{Personal}&\textbf{Formaci\'on acad\'emica}&\textbf{Funci\'on}&\textbf{Dedicaci\'on$/$   Valor mensual}&\textbf{Fundaci\'on del Banco de la Rep\'ublica} & \textbf{Contrapartida} \\ \hline
Investigador principal & D. Phil. & Desarrollo y direcci\'on del proyecto & 10 horas$/$semana & \multicolumn{1}{r|}{$\$$0} & \multicolumn{1}{r|}{$\$$15.953.700} \\ \hline
Estudiante de maestr\'ia &&Realizaci\'on de c\'alculos& 24 horas$/$semana $\$$1.500.000$/$mes por seis meses & \multicolumn{1}{r|}{$\$$9.000.000} & \multicolumn{1}{r|}{$\$$0} \\ \hline
\multicolumn{4}{|c|}{\text{TOTAL}} & \multicolumn{1}{r|}{$\$$9.000.000} & \multicolumn{1}{r|}{$\$$15.953.700} \\ \hline

\end{tabular}
\end{center}

\textbf{Presupuesto detallado para viaje de visita acad\'emica}\\

El presente proyecto se realizar\'a en colaboraci\'on con investigadores internacionales. La financiaci\'on de la Fundaci\'on del Banco de la Rep\'ublica permitir\'a realizar encuentros cient\'ificos con dichos colaboradoras para avanzar en el desarrollo del proyecto.\\

Objetivo del viaje: Discusi\'on de resultados de la investigaci\'on con colaboradores internacionales, escritura conjunta de art\'iculos cient\'ificos.\\
Duraci\'on de viaje: 20 d\'ias\\
Fuente de financiaci\'on: Fundaci\'on del Banco de la Rep\'ublica.\\

\begin{center}
\begin{tabular}{|c|c|}\hline
\textbf{Rubro} & \textbf{Costo}\\ \hline
Vuelos ida y vuelta & 4.000.000\\ \hline
Transporte en pa\'is de visita & 150.000\\ \hline
Alojamiento & 3.000.000\\ \hline
Alimentaci\'on & 1.500.000\\ \hline
Visa & 350.000\\ \hline
TOTAL & 9.000.000\\ \hline

\end{tabular}
\end{center}

\section{Impacto ambiental}
Dado que durante la ejecuci\'on del proyecto se llevar\'an a cabo principalmente actividades como b\'usqueda de bibliograf\'ia, simulaciones num\'ericas, c\'alculos anal\'iticos y divulgaci\'on de resultados, no se generar\'a un impacto ambiental negativo en su desarrollo. 

Por otro lado, la comprensi\'on de la f\'isica de superconductividad fuera del equilibrio, que es el \'area a la que el presente proyecto busca contribuir, tendr\'ia un impacto positivo bastante importante. El desarrollo de tecnolog\'ia basada en materiales con superconductividad estable a temperaturas cada vez m\'as altas permitir\'ia llevar a cabo tareas en que tal efecto es fundamental (por ejemplo a nivel m\'edico~\cite{aarnink2012euro} o investigativo~\cite{bottura2016ieee}), pero con un costo energ\'etico much\'isimo menor al requerido actualmente, pues no ser\'ia necesario inducir temperaturas muy bajas. Esto permitir\'ia un gran ahorro de los recursos usados para realizar procesos de enfriamiento a dichas temperaturas.

Las nuevas tecnolog\'ias basadas en este efecto de superconductividad tambi\'en tendr\'ian la ventaja de ser m\'as limpias que muchas de las tecnolog\'ias actuales. Por ejemplo, el transporte de energ\'ia el\'ectrica requerir\'ia menos esfuerzo al no ser necesario compensar las p\'erdidas por la disipaci\'on en los materiales conductores usados hoy en d\'ia. Similarmente, medios de transporte basados en levitaci\'on magn\'etica llevar\'ian a un menor uso de combustibles f\'osiles, constituyendo una alternativa de transporte m\'as limpia~\cite{wang2009ieee}. 

\section{Pertinencia social}
Debido a la naturaleza te\'orica de la investigaci\'on propuesta en el presente proyecto, \'esta no tendr\'ia un impacto social inmediato. Sin embargo, dada la enorme cantidad de aplicaciones que tendr\'ia la estabilizaci\'on de superconductividad a temperaturas cada vez mayores, se tendr\'ia un impacto social positivo en un largo plazo, siendo importante para diferentes actividades de la vida diaria. Por ejemplo, la implementaci\'on masiva de medios de transporte basados en levitaci\'on magn\'etica permitir\'ia una movilizaci\'on a muy altas velocidades~\cite{wang2009ieee}, siendo una alternativa eficiente a los medios m\'as usados en la actualidad. Por otro lado, se podr\'ian realizar de forma menos costosa y m\'as eficiente im\'agenes diagn\'osticas con fines m\'edicos~\cite{aarnink2012euro}. Se espera que una vez exista una comprensi\'on te\'orica completa de la f\'isica de la superconductividad a altas temperaturas (incluso a temperatura ambiente), se desarrollen muchas m\'as tecnolog\'ias de aplicaci\'on com\'un que puedan ser de alto beneficio para la sociedad.

\section{Aspectos \'eticos}
Este inciso no aplica a nuestro proyecto, dado que no realizamos experimentaci\'on en humanos o animales.

\section{Divulgaci\'on}
Del presente proyecto se establece un compromiso de dos art\'iculos cient\'ificos en revistas internacionales indexadas. El primero corresponde a la din\'amica de estados CDW sin forzamiento para diferentes repulsiones de Coulomb y hopping fijo. Solo recientemente se han realizado estudios similares para la din\'amica de estados antiferromagn\'eticos, en 1D~\cite{heidrich_meisner2015pra} y altas dimensiones~\cite{balzer2015prx,nosotros2017driven}. El an\'alisis de estados CDW sin forzamiento es por tanto novedoso, y constituir\'a un trabajo de alto impacto por s\'i mismo. Como se indica en el cronograma, este art\'iculo ser\'a sometido a los pocos meses de haber iniciado el proyecto. Por esto se hace el compromiso de tenerlo publicado al final del proyecto. El segundo art\'iculo corresponde a la respuesta de los estados CDW a los diferentes tipos de forzamiento peri\'odico, y a la determinaci\'on de existencia de estados superconductores. Dado que este trabajo se extiende hasta el final del a\~no, como se indica en el cronograma, se establece el compromiso de haberlo sometido a publicaci\'on para entonces (la aceptaci\'on y publicaci\'on se dar\'a unos pocos meses despu\'es). 

Durante el a\~no se realizar\'a divulgaci\'on del proyecto por otros medios. Estos incluyen la presentaci\'on de resultados en seminarios de f\'isica en diferentes universidades nacionales, y en conferencias nacionales e internacionales.

\section{Formaci\'on de recurso humano}
La aprobaci\'on de financiaci\'on para el presente proyecto corresponde a un compromiso de parte del investigador de formar un estudiante de maestr\'ia. Se entrenar\'a en los m\'etodos computacionales descritos en la propuesta, y en los diferentes conceptos te\'oricos abarcados en \'esta (din\'amica de fermiones, superconductividad, teor\'ia de Floquet y desigualdades de Leggett-Garg). Con este entrenamiento el estudiante no solamente aprender\'a a utilizar algunos de los m\'etodos m\'as poderosos para estudiar sistemas de materia condensada, sino que podr\'a desarrollar varios de los c\'alculos num\'ericos necesarios para cumplir con los objetivos presentados en el proyecto. El estudiante ser\'a coautor de los ar\'iculos para los cuales haya contribuido, y podr\'a recibir financiaci\'on para un semestre.    

\section{Hoja de vida de investigador}

La hoja de vida en el formato requerido del investigador principal se encuentra anexa al presente documento.

\section{Responsabilidades}

Responsabilidades del investigador principal:

\begin{itemize}
\item Revisi\'on de bibliograf\'{i}a
\item Desarrollo de c\'alculos anal\'{i}ticos y num\'ericos
\item Discusi\'on y an\'alisis de resultados obtenidos
\item Elaboraci\'on de reportes
\item Escritura de art\'{i}culos cient\'{i}ficos
\item Presentaci\'on de resultados en charlas y conferencias
\end{itemize}

%Responsabilidades de coinvestigadores:
%
%\begin{itemize}
%\item Desarrollo de c\'alculos anal\'{i}ticos y num\'ericos
%\item Discusi\'on y an\'alisis de resultados obtenidos
%\item Escritura de art\'{i}culos cient\'{i}ficos
%\item Presentaci\'on de resultados en charlas y conferencias
%\end{itemize}

\bibliographystyle{unsrtnt}

\bibliography{proyecto_BR_bib}

\end{document}

% ****** Start of file apssamp.tex ******
%
%   This file is part of the APS files in the REVTeX 4.1 distribution.
%   Version 4.1r of REVTeX, August 2010
%
%   Copyright (c) 2009, 2010 The American Physical Society.
%
%   See the REVTeX 4 README file for restrictions and more information.
%
% TeX'ing this file requires that you have AMS-LaTeX 2.0 installed
% as well as the rest of the prerequisites for REVTeX 4.1
%
% See the REVTeX 4 README file
% It also requires running BibTeX. The commands are as follows:
%
%  1)  latex apssamp.tex
%  2)  bibtex apssamp
%  3)  latex apssamp.tex
%  4)  latex apssamp.tex
%
\documentclass[%
 reprint,
%superscriptaddress,
%groupedaddress,
%unsortedaddress,
%runinaddress,
%frontmatterverbose, 
%preprint,
%showpacs,preprintnumbers,
%nofootinbib,
%nobibnotes,
%bibnotes,
 amsmath,amssymb,
 aps,
%pra,
%prb,
%rmp,
%prstab,
%prstper,
%floatfix,
]{revtex4-1}

\usepackage{graphicx}% Include figure files
\usepackage{dcolumn}% Align table columns on decimal point
\usepackage[spanish]{babel}
\selectlanguage{spanish} 
\usepackage[utf8]{inputenc}
\usepackage{bm}% bold math
%\usepackage{hyperref}% add hypertext capabilities
%\usepackage[mathlines]{lineno}% Enable numbering of text and display math
%\linenumbers\relax % Commence numbering lines
\usepackage{float}
%\usepackage[showframe,%Uncomment any one of the following lines to test 
%%scale=0.7, marginratio={1:1, 2:3}, ignoreall,% default settings
%%text={7in,10in},centering,
%%margin=1.5in,
%%total={6.5in,8.75in}, top=1.2in, left=0.9in, includefoot,
%%height=10in,a5paper,hmargin={3cm,0.8in},
%]{geometry}
\usepackage[font=footnotesize,labelfont=bf]{caption}
\usepackage{hyperref}
\newcommand{\subtitle}[1]{%
\posttitle{%
    \par\end{center}
\begin{center}\large#1\end{center}
\vskip0.5em}%
}
\begin{document}

%\preprint{APS/123-QED}

\title{Bombeo óptico\\ \textit{Estudio del efecto Zeeman fuerte y débil sobre un núcleo de Rubidio } }% Force line breaks with \\

%\subtitle{Estudio de la dualidad de onda partícula}

\author{Jose Alejandro Montaña Cortés}
\email{ja.montana@uniandes.edu.co}
% \altaffiliation[Also at ]{Departamento de Física, Universidad de los Andes}
\author{Jesús David Rincón Puche}%
\email{jd.rincon883@uniandes.edu.co}
\affiliation{Departamento de Física, Universidad de los Andes}%

%\collaboration{}%\noaffiliation

\date{\today}% It is always \today, today,
             %  but any date may be explicitly specified

\begin{abstract}



\end{abstract}
\maketitle
%\tableofcontents

%---------------------INTRODUCCIÓN------------------
\section{Introducción}
 
% trim={<left> <lower> <right> <upper>}
%----MONTAJE EXPERIMENTAL--------
\section{Montaje experimental}

%-----------------RESULTADOS----------------------
\section{Resultados y Análisis}


%---------------CONCLUSIONES-------------------

\section{Conclusiones}
\begin{itemize}
    \item 
\end{itemize}
%Se deben contestar las preguntas planteadas inicialmente o dar las razones por las cuales no es posible hacerlo. Las conclusiones deben ser necesariamente una consecuencia del experimento realizado, es decir que no se deben tocar aspectos que no se hayan expuesto en la sección de resultados y análisis. Si escribe algo que no se encuentra en la sección de resultados y análisis, esto quiere decir que hace falta incluir material en resultados y análisis. Concluir únicamente aspectos pertinentes a su trabajo en el laboratorio; evite generalizaciones que no hablan concretamente de lo que usted hizo o midió.

\begin{thebibliography}{9}
\bibitem{opticaondulatoria}
Université Moulay Ismaïl, 2017
\url{https://www.overleaf.com/project/5c58b017b206d66446cbc4bd}
\bibitem{MIT} 
Massachusetts Institute of Technology
\url{http://web.mit.edu/viz/EM/visualizations/coursenotes/modules/guide14.pdf}
\bibitem{guia} 
Inc. David A. Van Baak; TeachSpin. Two-slit interference one photon at a time \textit{The Essential Quantum Paradox}, instruction manual
Suite 409 Buffalo NY, 2013.
\url{http://physics-astronomy-manuals.wwu.edu/TeachSpin\%20Two\%20Slit\%20Expanded\%20Manual.pdf}
\end{thebibliography}

\end{document}
%
% ****** End of file apssamp.tex ******


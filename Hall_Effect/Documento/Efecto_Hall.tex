% ****** Start of file apssamp.tex ******
%
%   This file is part of the APS files in the REVTeX 4.1 distribution.
%   Version 4.1r of REVTeX, August 2010
%
%   Copyright (c) 2009, 2010 The American Physical Society.
%
%   See the REVTeX 4 README file for restrictions and more information.
%
% TeX'ing this file requires that you have AMS-LaTeX 2.0 installed
% as well as the rest of the prerequisites for REVTeX 4.1
%
% See the REVTeX 4 README file
% It also requires running BibTeX. The commands are as follows:
%
%  1)  latex apssamp.tex
%  2)  bibtex apssamp
%  3)  latex apssamp.tex
%  4)  latex apssamp.tex
%
\documentclass[%
 reprint,
%superscriptaddress,
%groupedaddress,
%unsortedaddress,
%runinaddress,
%frontmatterverbose, 
%preprint,
%showpacs,preprintnumbers,
%nofootinbib,
%nobibnotes,
%bibnotes,
 amsmath,amssymb,
 aps,
%pra,
%prb,
%rmp,
%prstab,
%prstper,
%floatfix,
]{revtex4-1}

\usepackage{graphicx}% Include figure files
\usepackage{dcolumn}% Align table columns on decimal point
\usepackage[spanish]{babel}
\selectlanguage{spanish} 
\usepackage[utf8]{inputenc}
\usepackage{bm}% bold math
%\usepackage{hyperref}% add hypertext capabilities
%\usepackage[mathlines]{lineno}% Enable numbering of text and display math
%\linenumbers\relax % Commence numbering lines
\usepackage{float}
%\usepackage[showframe,%Uncomment any one of the following lines to test 
%%scale=0.7, marginratio={1:1, 2:3}, ignoreall,% default settings
%%text={7in,10in},centering,
%%margin=1.5in,
%%total={6.5in,8.75in}, top=1.2in, left=0.9in, includefoot,
%%height=10in,a5paper,hmargin={3cm,0.8in},
%]{geometry}
\usepackage[font=footnotesize,labelfont=bf]{caption}
\usepackage{hyperref}
\newcommand{\subtitle}[1]{%
\posttitle{%
    \par\end{center}
\begin{center}\large#1\end{center}
\vskip0.5em}%
}
\begin{document}

%\preprint{APS/123-QED}

\title{Efecto Hall\\ \textit{Estudio de efectos clásicos y cuánticos } }% Force line breaks with \\

%\subtitle{Estudio de la dualidad de onda partícula}

\author{Jose Alejandro Montaña Cortés}
\email{ja.montana@uniandes.edu.co}
% \altaffiliation[Also at ]{Departamento de Física, Universidad de los Andes}
\author{Jesús David Rincón Puche}%
\email{jd.rincon883@uniandes.edu.co}
\affiliation{Departamento de Física, Universidad de los Andes}%

%\collaboration{}%\noaffiliation

\date{\today}% It is always \today, today,
             %  but any date may be explicitly specified

\begin{abstract}

En este experimento se realizaron medidas para el efecto Hall clásico sobre una placa semiconductora tipo P de germanio puro. Primero se realizó una calibración del campo magnético que actúa sobre la placa, encontrando que aproximadamente 1 $A$ equivale a 200 $mT$. Seguido de esto, se realizaron medidas del voltaje Hall ($V_H$) en función de la corriente paralela ($I_p$) y se encuentra exitosamente el comportamiento lineal esperado; en adición,se observó que al agregar campo magnético el valor de la resistencia de Hall ($R_H$) disminuye, gracias al aumento en $V_H$. En la tercera parte, se realizó un experimento para observar la magnetoresistencia, fenómeno en el cual una resistencia eléctrica aumenta gracias a un campo magnético aplicado, sin mucho éxito por posibles daños en la placa semiconductora. Finalmente, se desarrolló un experimento para observar como la conductividad disminuye con la temperatura, obteniendo resultados deseados para este experimento.

\end{abstract}
\maketitle
%\tableofcontents

%---------------------INTRODUCCIÓN------------------
\section{Introducción}

 En 1879, Edwin Hall observó que cuando una corriente eléctrica pasa a través de una muestra confinada en un campo magnético, un potencial eléctrico, proporcional tanto a la corriente como a dicho campo, es generado en la dirección perpendicular a estos. Este fenómeno es lo que se denomina como efecto Hall y tiene muchas aplicaciones, principalmente para equipos electrónico, como audífonos o motores eléctricos~\cite{Efecto Hall}. Para este experimento, se planea ver el efecto Hall actuar sobre una placa de dopaje p, el cuál se explicará en la siguiente sección, junto a otros efectos que cambian la naturaleza del efecto Hall, como el efecto Ettinghause, aumentando las colisiones en el sistema; el efecto Nernst, reduciendo la resistencia Hall; y el efecto Righic Leduc, reduciendo la conductividad del sistema con un aumento de temperatura (estos efectos se explicarán más adelante).

 \section{Desarrollo Teórico}
 
 \subsection{Dopajes tipo n y p}
 
 
 
% trim={<left> <lower> <right> <upper>}
%----MONTAJE EXPERIMENTAL--------
\section{Montaje experimental}
\subsection{Instrumento de medición.}


%-----------------RESULTADOS----------------------
\section{Resultados y Análisis}
\subsection{Relación entre la corriente que pasa por el montaje y el campo magnético generado}
Dado que el Teslámetro que se tenía a disposición era un aparato sensible y delicado, se empezó primero por caracterizar como se comportaba el campo magnético generado por las bobinas en función de la corriente que circulaba por estas. Esto con el fin de que cuando se hablara de campo magnético, se supiera su equivalente en términos de la corriente y así con esto minimizar el uso del aparato.\\
Tal y como se muestra en la figura \ref{Calibracion}, la relación entre la corriente que circula por las bobinas y el campo magnético es de tipo lineal. Con esto se tiene entonces que hablar de corriente es equivalente a hablar de campo magnético\footnote{Es por esto que de ahora en adelante cuando se hable de campo magnético sus unidades serán Amperios}.

\begin{figure}[h]
\center{\includegraphics[width=0.4\textwidth]
{../Figuras/Calibracion_error.png}}
\caption{\label{Calibracion}. Gráfico del campo magnético medido por el Teslametro en función de la corriente suministrada por la fuente. La linea en azul corresponde al ajuste efectuado, a los datos experimentales y la región sombreada corresponde a los intervalos de confidencia con $5\sigma$}
\end{figure}

\subsection{Determinación de la resistencia de Hall.}
Una vez caracterizado el campo magnético en función de la corriente, se fijó el campo magnético en 1.5 A y se analizó el comportamiento entre el voltaje de Hall $V_H$ en función de la corriente que pasaba por el material $I_p$.\\
Tal y como se muestra en la figura \ref{V_H_vs_I_p} se tiene que la relación entre estas dos cantidades es de tipo lineal y a partir de la regresión se tiene entonces que el valor de la resistencia de Hall es:
\[R_H=\frac{1.46\cdot 10^{-3}}{1.5(164.62)+17.05}=5.53\times 10^{-3} \pm (2.9\times 10^{-4}) \Omega\]
\begin{figure}[h]
\center{\includegraphics[width=0.4\textwidth]
{../Figuras/Voltaje_hall_ip.png}}
\caption{\label{V_H_vs_I_p}Voltaje de Hall en función de la corriente que atraviesa la placa de tipo $P$, Se observó una tendencia de tipo lineal para los datos experimentales. La linea en azul corresponde al ajuste efectuado, a los datos experimentales y la región sombreada corresponde a los intervalos de confidencia con $5\sigma$}
\end{figure}

 De forma similar a la parte anterior se fijó el valor de la corriente $I_p$ en $-30 mA$ y luego se procedió a variar el campo magnético. En la figura \ref{V_H_vs_campo}  se observa el patrón obtenido, lo observado es que al igual que para la figura \ref{V_H_vs_I_p} se observó un comportamiento de tipo lineal, tal y como se esperaba, salvo que para este caso la linea formada tiene pendiente negativa, esto es explicado por el hecho de que la corriente que pasaba por la muestra tenía signo negativo.
\begin{figure}[h]
\center{\includegraphics[width=0.4\textwidth]
{../Figuras/Voltaje_hall_Campo_magnetico.png}}
\caption{\label{V_H_vs_campo}.  La linea en azul corresponde al ajuste efectuado, a los datos experimentales y la región sombreada corresponde a los intervalos de confidencia con $5\sigma$}
\end{figure}

\begin{figure}[h]
\center{\includegraphics[width=0.4\textwidth]
{../Figuras/Voltaje_longitudinal_ip.png}}
\caption{\label{V longitudinal vs ip}.  La linea en azul corresponde al ajuste efectuado, a los datos experimentales y la región sombreada corresponde a los intervalos de confidencia con $5\sigma$}
\end{figure}

\begin{figure}[h]
\center{\includegraphics[width=0.4\textwidth]
{../Figuras/Resistencia_vs_Campo_magnetico.png}}
\caption{\label{R_vs_Campo}.  La linea en azul corresponde al ajuste efectuado, a los datos experimentales y la región sombreada corresponde a los intervalos de confidencia con $5\sigma$}
\end{figure}



\begin{figure}[h]
\center{\includegraphics[width=0.4\textwidth]
{../Figuras/Comnductividad_Temperatura.png}}
\caption{\label{Conductividad en función de la temperatura}.  La linea en azul corresponde al ajuste efectuado, a los datos experimentales y la región sombreada corresponde a los intervalos de confidencia con $5\sigma$}
\end{figure}


\begin{figure}[h]
\center{\includegraphics[width=0.4\textwidth]
{../Figuras/Region_intrinseca.png}}
\caption{\label{Region intrinseca}.  La linea en azul corresponde al ajuste efectuado, a los datos experimentales y la región sombreada corresponde a los intervalos de confidencia con $5\sigma$}
\end{figure}


\begin{figure}[h]
\center{\includegraphics[width=0.4\textwidth]
{../Figuras/Movilidad_Hall.png}}
\caption{\label{movilidad de Hall}.  La linea en azul corresponde al ajuste efectuado, a los datos experimentales y la región sombreada corresponde a los intervalos de confidencia con $5\sigma$}
\end{figure}
%---------------CONCLUSIONES-------------------

\section{Conclusiones}
\begin{itemize}
    \item 
    \item 
\end{itemize}
%Se deben contestar las preguntas planteadas inicialmente o dar las razones por las cuales no es posible hacerlo. Las conclusiones deben ser necesariamente una consecuencia del experimento realizado, es decir que no se deben tocar aspectos que no se hayan expuesto en la sección de resultados y análisis. Si escribe algo que no se encuentra en la sección de resultados y análisis, esto quiere decir que hace falta incluir material en resultados y análisis. Concluir únicamente aspectos pertinentes a su trabajo en el laboratorio; evite generalizaciones que no hablan concretamente de lo que usted hizo o midió.

\begin{thebibliography}{9}

\bibitem{Efecto Hall}
Woodford, c. \textit{Hall Effect Sensors}, \textit{Explain that stuff}, 2018 \url{https://www.explainthatstuff.com/hall-effect-sensors.html}
\bibitem{Klitzing}
 K. v. Klitzing, G. Dorda, and M. Pepper. \textit{New method for high-accuracy determination of thefine-structure constant based on quantized hall resistance}.Phys. Rev. Lett., 45:494–497, Aug 198
\bibitem{figura_aparato}
\url{http://spa-mxpweb.spa.umn.edu/s11/Projects/S11_OpticalPumping/apparatus.htm}
\bibitem{guia optical pumping}
Mejía, J; Universidad de los Andes. \textit{Bombeo óptico}, 2017.

\end{thebibliography}

\end{document}
%
% ****** End of file apssamp.tex ******

